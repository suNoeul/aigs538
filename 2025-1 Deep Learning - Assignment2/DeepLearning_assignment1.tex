\documentclass{article}

\usepackage[final]{neurips_2023}

\usepackage[utf8]{inputenc} % allow utf-8 input
\usepackage[T1]{fontenc}    % use 8-bit T1 fonts
\usepackage{hyperref}       % hyperlinks
\usepackage{url}            % simple URL typesetting
\usepackage{booktabs}       % professional-quality tables
\usepackage{amsfonts}       % blackboard math symbols
\usepackage{nicefrac}       % compact symbols for 1/2, etc.
\usepackage{microtype}      % microtypography
\usepackage{xcolor}         % colors
\usepackage{natbib}         % bibliography


\title{Deep Learning - Assignment2}



% The \author macro works with any number of authors. There are two commands
% used to separate the names and addresses of multiple authors: \And and \AND.
%
% Using \And between authors leaves it to LaTeX to determine where to break the
% lines. Using \AND forces a line break at that point. So, if LaTeX puts 3 of 4
% authors names on the first line, and the last on the second line, try using
% \AND instead of \And before the third author name.


\author{%
Name \\
Team \# - studentID\\
}


\begin{document}


\maketitle


\begin{abstract}
  Each student should write \textbf{one summary paper} by reading papers related to their final project.
  The submission deadline is \textbf{25th of April, 2025, 23:59}.
  For your literature survey, topic and content of the paper should be included with \textbf{1 page} length.
  The chosen papers should not overlap for students in the same team.
  The template should \underline{follow the current Latex file (originally from NeurIPS 2023 template)}, and you should not change the format, e.g., font size, margin size, any other format that affect the length of your report. For any additional citations/references, please refer to the .bib file to include it on the reference page.
  Also, the assignment should be written in \textbf{English}.
  \\\\Please change the title (Deep Learning - Project Assignment2) to the topic of your chosen paper.
  Your literature survey should contain the following contents.
  \textbf{1)} Topic,
  \textbf{2)} Problem Definition,
  \textbf{3)} Proposed Method,
  \textbf{4)} How your project is different from or influenced by this paper. \underline{Remember to include a reference for your choice of paper}, including title, venue, and publication year, on the reference page. \\
  

  This abstract is \underline{only for the instruction of the assignment.}
  \textit{Please remove the abstract section for your report.}
\end{abstract}

\section{Problem Definition}
The problem that the chosen paper\cite{samplepaper} is trying to handle.


\section{Proposed Method}
The method the chosen paper\cite{samplepaper} propose.

\section{Discussion}
How your project is different from or influenced by this paper\cite{samplepaper}.


\newpage
\textit{ERASE THIS SECTION BELOW BEFORE SUBMISSION}\\
References follow the acknowledgments in the camera-ready paper. Use unnumbered first-level heading for
the references. Any choice of citation style is acceptable as long as you are
consistent. It is permissible to reduce the font size to \verb+small+ (9 point)
when listing the references.
Note that the Reference section does not count towards the page limit.

\bibliographystyle{plain}
\bibliography{reference}
\end{document}