\documentclass{article}

\usepackage[final]{neurips_2023}

\usepackage[utf8]{inputenc} % allow utf-8 input
\usepackage[T1]{fontenc}    % use 8-bit T1 fonts
\usepackage{hyperref}       % hyperlinks
\usepackage{url}            % simple URL typesetting
\usepackage{booktabs}       % professional-quality tables
\usepackage{amsfonts}       % blackboard math symbols
\usepackage{nicefrac}       % compact symbols for 1/2, etc.
\usepackage{microtype}      % microtypography
\usepackage{xcolor}         % colors


\title{Deep Learning - Project Proposal}



% The \author macro works with any number of authors. There are two commands
% used to separate the names and addresses of multiple authors: \And and \AND.
%
% Using \And between authors leaves it to LaTeX to determine where to break the
% lines. Using \AND forces a line break at that point. So, if LaTeX puts 3 of 4
% authors names on the first line, and the last on the second line, try using
% \AND instead of \And before the third author name.


\author{%
Seungjoo Lee \\
20242700\\
\And
Minjae Lee \\
20232244\\
\And
Jehyeon Shin \\
20232457\\
\And
Student4 Name \\
ID
  % examples of more authors
  % \And
  % Coauthor \\
  % Affiliation \\
  % Address \\
  % \texttt{email} \\
  % \AND
  % Coauthor \\
  % Affiliation \\
  % Address \\
  % \texttt{email} \\
  % \And
  % Coauthor \\
  % Affiliation \\
  % Address \\
  % \texttt{email} \\
}


\begin{document}


\maketitle


\begin{abstract}
  The project proposal for the final project should be submitted by the \textbf{3rd of April, 2025}.
  The template should follow the current Latex file (originally from NeurIPS 2023 template), and you should not change the format, e.g., font size, margin size, any other format that affect the length of your report.
  The length of your proposal should be \textbf{1 page}, with an additional 1 page for references.
  (That means your writings should be at most 1 page.)
  Also, the proposal should be written in \textbf{English}.
  Please change the title (Deep Learning - Project Proposal) as your own topic/title.
  Your proposal should contain the following contents.
  \textbf{0)} Title,
  \textbf{1)} Background / Related Work
  \textbf{2)} Problem Definition 
  \textbf{3)} Your Idea/Method,
  \textbf{4)} Any other contents/comments that the Instructor/TA should know.
  

  This abstract is only for the instruction of the project proposal.
  \textit{Please remove the abstract section for your report.}
\end{abstract}

\section{Background/Related Work}
This section should contain previous works, i.e., previous approach or related works to the task. The subsections below are just given as examples.

\section{Problem Definition}
Problem definition should include the motivation of your chosen topic.
You can choose any deep learning related topic for your project.
However, please write your proposal as precise as possible, as some topics may not be very familiar or well known.



\subsection{Conventional Methods}
Conventional methods dealt the stated problem by @@@.

\subsection{State-of-the-art Methods}
The state-of-the-art methods in this task are @@@.

\section{Proposed Idea/Method}
This section should contain your own method.
As this is the "proposal", you don't have to have experimental results or settings here.




\newpage
\section*{References}


References follow the acknowledgments in the camera-ready paper. Use unnumbered first-level heading for
the references. Any choice of citation style is acceptable as long as you are
consistent. It is permissible to reduce the font size to \verb+small+ (9 point)
when listing the references.
Note that the Reference section does not count towards the page limit.
\medskip


{
\small


[1] Alexander, J.A.\ \& Mozer, M.C.\ (1995) Template-based algorithms for
connectionist rule extraction. In G.\ Tesauro, D.S.\ Touretzky and T.K.\ Leen
(eds.), {\it Advances in Neural Information Processing Systems 7},
pp.\ 609--616. Cambridge, MA: MIT Press.


[2] Bower, J.M.\ \& Beeman, D.\ (1995) {\it The Book of GENESIS: Exploring
  Realistic Neural Models with the GEneral NEural SImulation System.}  New York:
TELOS/Springer--Verlag.


[3] Hasselmo, M.E., Schnell, E.\ \& Barkai, E.\ (1995) Dynamics of learning and
recall at excitatory recurrent synapses and cholinergic modulation in rat
hippocampal region CA3. {\it Journal of Neuroscience} {\bf 15}(7):5249-5262.
}

%%%%%%%%%%%%%%%%%%%%%%%%%%%%%%%%%%%%%%%%%%%%%%%%%%%%%%%%%%%%


\end{document}